\documentclass[a4paper,11pt]{article}
\usepackage[utf8]{inputenc}
\usepackage[spanish]{babel}
\usepackage[hmargin=3cm, vmargin=3cm]{geometry}
\usepackage{graphicx}
\usepackage{amssymb}
\usepackage{amsmath}
\usepackage{enumitem}

% Define counter and exercise titles
\newcounter{exercise}
\newcommand{\exercisetitle}[1]{
    \stepcounter{exercise}
    \vspace{1em}
    \noindent{
        \textbf{Problema \theexercise . #1}
        \vspace{0.5em}
    }
}


%%%%%%%%%%%%%%%%%%%%%%%%%%%%%%%%%%%%%%%%%%%%%%%%%%%%%%%%%%%%%%%%%%%%%%%%%%%%%%

\begin{document}

\begin{center}
    \textbf{\large
        Mecánica Estadística \\
        Práctica 1 - Colectivo Microcanónico
    }
\end{center}


\exercisetitle{}

 Suponiendo que la entropía $S$ y la cantidad de microestados 
$\Omega$ de un sistema físico están relacionadas por una función 
arbitraria $f$ tal que:
~
$$ S = f(\Omega), $$
~
\noindent demuestre que el caracter extensivo de $S$ y el multiplicativo de 
$\Omega$ necesariamente requieren que $f$ sea de la forma:
~
$$ S = k \ln \Omega $$
~


\exercisetitle{Aproximación de Stirling}

Mostrar que $\ln N! = \sum_{n=0}^N \ln n$.
Aproximando esta suma por una integral, obtener la aproximación de 
stirling:
~
$$ \ln N! \simeq N \ln N - N $$



\exercisetitle{Paramagneto}

Una sustancia paramagnética es aquella que presenta magnetización 
$\mathbf{M}$ nula frente a un campo magnético externo $\mathbf{H}$ 
nulo.
Si por el contrario, el campo no es nulo, el paramagneto se 
imanta en la misma dirección. Se pueden considerar dos calses de 
paramagnetismo: de \emph{Langevin} o de \emph{Pauli}.
El segundo se debe al gas de electrones libres de un metal, mientras 
que el primero se origina en la interacción de los momentos magnéticos 
intrínsecos de los átomos del sólido con el campo magnético externo.
En este problema analizaremos un caso sencillo del magnetismo de 
Langevin.

Supongamos un paramagneto inmerso en un campo magnético $H$ 
orientado en la dirección del eje $z$.
Consideremos que el paramagneto está compuesto por $N$ momentos 
magnéticos individuales (y no interactuantes entre sí) de manera tal 
que la proyección de cada momento magnético en el eje $z$ 
puede ser $\pm \mu_B$ ($\mu_B = e\hbar / 2m_e c = 9.273 \times 
10^{-24} \text{J/T}$ se conoce como el magnetón de Bohr).
La magnetización total del sistema estará dada por $M = m \mu_B$, donde 
$N/2 + m/2$ y $N/2 - m/2$ son las cantidades de momentos magnéticos que 
apuntan hacia arriba y hacia abajo, respectivamente.
Vamos a considerar también que el sistema ``paramagneto + campo 
magnético'' es un sistema aislado, es decir, $E = - 
\mathbf{M}\cdot\mathbf{H} = \text{cte}$.

\begin{enumerate}[label=(\alph*),
                  leftmargin=2\parindent,
                  rightmargin=2\parindent]

    \item{Encontrar una expresión analítica para el número de 
          estados microscópicos $\Omega(N, m)$ correspondientes a una 
          magnetización total M.
          }
    
    \item{\label{item:paramagneto-magnetizacion}
          Usando la relación
          $\frac{1}{T} = \left( \frac{\partial S}{\partial E} \right)_V$,
          hallar la ecuación de estado de los paramagnetos:
          $$ M = N \mu_B \tanh \left( \frac{\mu_B H}{k_B T} \right) $$.
          }
    
    \item{\label{item:paramagneto-grafica}
          Realizar una gráfica de $M$ vs $\mu_B H/k_B T$.
          ¿Qué sucede con la magnetización a altas temperaturas?
          ¿Qué sucede con la magnetización a bajas temperaturas?
          }
    
    {\small
    \textbf{Ayuda:} Cuando hablamos de bajas temperaturas nos referimos 
    a que $k_B T \ll \mu_B H$, mientras que altas temperaturas 
    significa que $k_B T \gg \mu_B H$.
    }

    \item{La Ley de Curie (obtenida experimentalmente por Pierre Curie en 1896)
          establece que la magnetización de un paramagneto es 
          directamente proporcional al campo magnético externo e 
          inversamente proporcional a la temperatura:
          $$ M \propto \frac{H}{T} $$
          Determinar el rango de validez de la Ley de Curie a partir de la
          gráfica realizada en \ref{item:paramagneto-grafica}
          (¿altas o bajas temperaturas?).\\
          Obtener la Ley de Curie a partir de un desarrollo en series 
          de Taylor de la magnetización obtenida en 
          \ref{item:paramagneto-magnetizacion}.
          }
    
    \item{Si consideramos un paramagneto compuesto por $N = 10^{24}$ 
          momentos magnéticos a $T = 300 \rm{^\circ K}$ y sometido a un 
          campo magnético externo $H = 10^{-3} T$,
          ¿cuál es la energía necesaria para invertir totalmente la 
          magnetización $M$?
          }

\end{enumerate}



\exercisetitle{Sólido de Einstein}

Uno de los primeros modelos que se propusieron para describir las vibraciones 
de un sólido cristalino es el de Einstein. El mismo considera cada uno de los 
$N$ átomos del sólido vibrando alrededor de su posición de equilibrio y de 
manera independiente. Además, presupone que todos los átomos oscilan con la 
misma frecuencia $\omega$. Si consideramos a cada átomo como un oscilador 
harmónico cuántico 3D, cada uno posee niveles de energía discretos
($n = 1, 2, 3, \dots$) y por ende, la energía total del sistema ($E$) puede 
distribuirse entre los distintos osciladores en $E/\hbar\omega$ 
cuantos de energía.

\begin{enumerate}[label=(\alph*),
                  leftmargin=2\parindent,
                  rightmargin=2\parindent]

    \item{Calcular la cantidad de microestados $\Omega(E, N, V)$:\\
          $$\Omega(E, N, V) =
          \frac{(3N - 1 + E/\hbar\omega)!}{(3N - 1)! \, (E/\hbar\omega)!}$$
          }

    \item{Obtener una expresión para la energía total $E$ en función 
          de la temperatura y analizar los límites a la temperatura del 
          cero absoluto y a temperaturas altas ($T \rightarrow \infty$).
          }

    \item{Calcular el calor específico del sólido de Einstein.
          Analizar los casos de $T=0 \text{K}$ y $T \rightarrow \infty$.
          Realizar una gráfica del $C_v$ en función de $T$
          (utilizar Software como wxMaxima).
          }
\end{enumerate}

{\small
\textbf{Ayuda:} Al considerar cada átomo como un oscilador 3D, podemos asumir
que el sólido de $N$ átomos es equivalente a un sistema de $3N$ osciladores
harmónicos 1D. Además, podemos realizar un corrimiento de la energía total
del sistema y de esta manera considerar nula la energía de punto cero del
oscilador.
}

\end{document}

