\documentclass[a4paper,11pt]{article}
\usepackage[utf8]{inputenc}
\usepackage[spanish]{babel}
\usepackage[hmargin=3cm, vmargin=3cm]{geometry}
\usepackage{graphicx}
\usepackage{amssymb}
\usepackage{amsmath}
\usepackage{enumitem}

% Define counter and exercise titles
\newcounter{exercise}
\newcommand{\exercisetitle}[1]{
    \stepcounter{exercise}
    \vspace{0.5em}
    \noindent{
        \textbf{Problema \theexercise . #1}
        \vspace{0.5em}
    }
}


%%%%%%%%%%%%%%%%%%%%%%%%%%%%%%%%%%%%%%%%%%%%%%%%%%%%%%%%%%%%%%%%%%%%%%%%%%%%%%

\begin{document}

\begin{center}
    \textbf{\large
        Mecánica Estadística \\
        Práctica 1 - Colectivo Microcanónico
    }
\end{center}


\exercisetitle{}

 Suponiendo que la entropía $S$ y la cantidad de microestados 
$\Omega$ de un sistema físico están relacionadas por una función 
arbitraria $f$ tal que:

$$ S = f(\Omega), $$

\noindent demuestre que el caracter extensivo de $S$ y el multiplicativo de 
$\Omega$ necesariamente requieren que $f$ sea de la forma:

$$ S = k \ln \Omega $$



\exercisetitle{Sólido de Einstein}

Uno de los primeros modelos que se propusieron para describir las vibraciones 
de un sólido cristalino es el de Einstein. El mismo considera cada uno de los 
$N$ átomos del sólido vibrando alrededor de su posición de equilibrio y de 
manera independiente. Además, presupone que todos los átomos oscilan con la 
misma frecuencia $\omega$. Si consideramos a cada átomo como un oscilado 
harmónico cuántico, cada uno posee niveles de energía discretos ($n = 1, 2, 3, 
\dots$) y por ende, la energía total del sistema ($E$) puede 
distribuirse entre los distintos osciladores en $E/\hbar\omega$ 
cuantos de energía.

\begin{enumerate}[label=(\alph*),
                  leftmargin=2\parindent,
                  rightmargin=2\parindent]

    \item{Calcular la cantidad de microestados $\Omega(E, N, V)$:\\
          $$\Omega(E, N, V) =
          \frac{(3N - 1 + E/\hbar\omega)!}{(3N - 1)! \, (E/\hbar\omega)!}$$
          }

    \item{Obtener una expresión para la energía total $E$ en función 
          de la temperatura y analizar los límites a la temperatura del 
          cero absoluto y a temperaturas altas ($T \rightarrow \infty$).
          }

    \item{Calcular el calor específico del sólido de Einstein.
          Analizar los casos de $T=0$ y $T \rightarrow \infty$.
          Realizar una gráfica del $C_v$ en función de $T$
          (utilizar Software como wxMaxima).
          }
\end{enumerate}


\end{document}

