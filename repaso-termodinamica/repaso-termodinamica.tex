\documentclass[a4paper,11pt]{article}
\usepackage[utf8]{inputenc}
\usepackage[spanish]{babel}
\usepackage[hmargin=3cm, vmargin=3cm]{geometry}
\usepackage{graphicx}
\usepackage{amssymb}
\usepackage{amsmath}
\usepackage{enumitem}
\usepackage{subcaption}
\usepackage{fancyhdr}
\usepackage{titlesec}


\titleformat{\section}
  {\bf}{Problema \thesection.}{0.5em}{}


%%%%%%%%%%%%%%%%%%%%%%%%%%%%%%%%%%%%%%%%%%%%%%%%%%%%%%%%%%%%%%%%%%%%%%%%%%%%%%

\begin{document}


% Fancy Header
% ------------
\pagestyle{fancy}
%~ \renewcommand{\headrulewidth}{0pt}
\lhead{\small Veronica Gargiulo}
\chead{\small \the\year}
\rhead{\small Santiago Soler}



% Title
% -----
\thispagestyle{plain}
\begin{center}
    \textbf{\large
        Mecánica Estadística \\
        Repaso de Termodinámica
    }
\end{center}
\vspace{-1.5em}



% Excersises
% ----------
\section{}

Consideremos $n$ moles de un gas ideal monoatómico que se encuentra en 
un estado inicial de presión $P_A$ y volumen $V_A$. Supongamos que se 
incrementa la temperatura a volumen constante hasta duplicar la 
presión. Luego el gas se expande isotérmicamente hasta que la presión 
desciende a su valor original y posteriormente se comprime a presión 
constante hasta que el volumen recupera su valor inicial.

\begin{enumerate}[label=(\alph*),
                  leftmargin=2\parindent,
                  rightmargin=2\parindent]

    \item{Representar estos procesos en el plano $P$-$V$ y en el plano 
          $P$-$T$.}
    
    \item{Calcular el trabajo realizado por el sistema, el calor 
          entregado al mismo y su variación de la energía interna en 
          cada proceso.}
    
    \item{Calcular la variación energía interna del ciclo completo e 
          interpretar el resultado.}
    
    \item{Calcular el trabajo realizado por el sistema y el calor 
          entregado al mismo a lo largo del ciclo completo. ¿El sistema 
          realiza trabajo o se realiza trabajo sobre el mismo? ¿El 
          sistema recibe calor o lo entrega al entorno?}

\end{enumerate}


\section{}

Consideremos dos bloques $A$ y $B$ del mismo metal que se encuentran 
inicialmente a temperaturas $T_A$ y $T_B$, respectivamente, tal que 
$T_A > T_B$. Dichos bloques se ponen en contacto dentro de un 
recipiente con paredes adiabáticas hasta que ambos alcanzan la misma 
temperatura $T_f$.

\begin{enumerate}[label=(\alph*),
                  leftmargin=2\parindent,
                  rightmargin=2\parindent]

    \item{Analizando cualitativamente el proceso que sufren ambos 
          bloques, ¿será reversible?}
    
    \item{Si el calor específico del metal es $C_V = 3 N k_B$,
          donde $N = 10^{23}$ y $k_B$ es la constante de Boltzmann, 
          calcular la variación de entropía de cada bloque a lo largo 
          del proceso y la del sistema $A \cup B$. 
          Teniendo en cuenta este último resultado, ¿el proceso es 
          reversible?}

\end{enumerate}

\end{document}

