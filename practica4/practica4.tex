\documentclass[a4paper,11pt]{article}
\usepackage[utf8]{inputenc}
\usepackage[spanish]{babel}
\usepackage[hmargin=3cm, vmargin=3cm]{geometry}
\usepackage{graphicx}
\usepackage{amssymb}
\usepackage{amsmath}
\usepackage{braket}
\usepackage{enumitem}
\usepackage{subcaption}
\usepackage{fancyhdr}
\usepackage{titlesec}


\titleformat{\section}
  {\bf}{Problema \thesection.}{0.5em}{}


%%%%%%%%%%%%%%%%%%%%%%%%%%%%%%%%%%%%%%%%%%%%%%%%%%%%%%%%%%%%%%%%%%%%%%%%%%%%%%

\begin{document}


% Fancy Header
% ------------
\pagestyle{fancy}
%~ \renewcommand{\headrulewidth}{0pt}
\lhead{\small Veronica Gargiulo}
\chead{\small \the\year}
\rhead{\small Santiago Soler}



% Title
% -----
\thispagestyle{plain}
\begin{center}
    \textbf{\large
        Mecánica Estadística \\
        Práctica 4 - Gases Cuánticos
    }
\end{center}
\vspace{-1.5em}



% Excersises
% ----------

\section{Gas de Fermiones}
\label{sec:gas-de-fermi}

Consideremos un gas de fermiones no interactuantes dentro de un 
recipiente de volumen $V$.
Dada la naturaleza de los fermiones, debemos tratar al gas bajo una 
estadística de Fermi-Dirac, a partir de la cual podemos determinar el 
número medio de ocupación de cada nivel:

$$ n(\epsilon) = \frac{1}{e^{\beta(\epsilon - \mu)} + 1} $$

\noindent y la gran función de partición:

$$
\mathcal{Q}_{FD} =
  \prod_j \left[ e^{-\beta(\epsilon_j - \mu)} + 1\right],
$$

\noindent donde $\mu$ es el potencial químico del gas.

En el límite de $T = 0$ todos los estados de energía entre 0 y $\mu$ 
están ocupados, mientras que los niveles superiores se encuentran 
disponibles.
En esta situación se considera que el gas está \emph{completamente 
degenerado}.
Dado este estado particular que presentan los gases de fermiones, se 
define a la energía de Fermi $\epsilon_F$ como:

$$ \epsilon_F = \mu(T=0). $$

\begin{enumerate}[label=(\alph*),
                  leftmargin=2\parindent,
                  rightmargin=2\parindent]
    
    \item{Determine el valor medio de número de partículas en el gas 
          ($N$) en el caso de un gas de Fermi completamente degenerado.
          Obtenga una expresión de la energía de Fermi en función de 
          $N$ y $V$, y demuestre que la densidad de estados del gas 
          puede escribirse como:
          $$
          g(\epsilon) =
            \frac{3}{2} \frac{N}{\epsilon_F} \left( 
            \frac{\epsilon}{\epsilon_F} \right)^{\frac{1}{2}}
          $$
          }

    \item{Determine el valor medio de la energía interna ($E$) para el 
          mismo gas completamente degenerado.
          }
    
    \item{Obtenga una expresión de la presión del gas de Fermi completamente 
          degenerado en función de la energía de Fermi.
          Demuestre que, al igual que en el caso del gas ideal 
          clásico, se cumple la siguiente relación entre la presión 
          y la energía interna:
          $$ pV = \frac{2}{3} \langle E \rangle. $$
          Resaltar el hecho de que incluso a $T = 0$, el gas 
          de Fermi presenta presiones no nulas.
          }

    \item{\textbf{Calor específico.}\\
          Para determinar el calor específico del gas de Fermi, al 
          menos en el rango de muy bajas temperaturas, es necesario 
          salirnos del límite de $T = 0$ para así poder obtener una 
          expresión de la energía media en función de $T$.
          La forma correcta de hacerlo sería resolver la integral que 
          involucra la energía interna considerando la expresión analítica 
          del número medio de ocupación $n(\epsilon)$, método que se 
          conoce como el desarrollo de Sommerfeld (1928).
          Sin embargo esto requiere mucho trabajo y se escapa de los 
          objetivos del ejercicio.

          Nosotros intentaremos una forma alternativa\footnote{Este 
          método alternativo se puede encontrar en el libro 
          \emph{Introduction to Solid State Physics}, Kittel, p.151.}, 
          la cual nos permitirá obtener al menos el primer término del 
          desarrollo de Sommerfeld.
          
          Si definimos $\Delta E = E(T) - E(T=0)$, el calor específico 
          puede calcularse como:
          $$
          C_V =
          \left( \frac{\partial \Delta E}{\partial T} \right)_{N, V},
          $$
          ya que restarle la energía del estado fundamental es 
          equivalente a un corrimiento rígido de la energía.
          
          Suponiendo que la cantidad de partículas en el gas no varía 
          significativamente para temperaturas cercanas al cero 
          absoluto, la diferencia entre $N(T)$ y $N(T=0)$ es 
          aproximadamente nula, y por ende puede ser introducida en la 
          expresión de $\Delta E$ (siendo previamente multiplicada por 
          $\epsilon_F$).
          Esta nueva expresión de $\Delta E$ es posible derivarla con 
          respecto a la temperatura para obtener una expresión del 
          calor específico.
          
          Demostrar entonces, que a través de este método obtenemos 
          el primer término del desarrollo de Sommerfeld:
          $$ C_V \simeq N k \frac{\pi^2}{2} \frac{kT}{\epsilon_F}. $$
         }
     
    \item{La capacidad de conducir electricidad que poseen los metales se 
          puede explicar gracias a la presencia de electrones libres en la 
          red cristalina, los cuales puede trasladarse dentro del metal casi 
          sin restricciones.
          Un método válido para modelizar estos electrones es 
          considerarlos como electrones libres dentro de un pozo de 
          potencial infinito, es decir, como un gas de fermiones no 
          interactuantes.
          Dado que la masa de un electrón es de $9.109 \cdot 
          10^{-38}$kg y dentro de un metal poseen una densidad de 1 
          nm$^{-3}$, calcule la temperatura de Fermi
          ($T_F = \epsilon_F/k_B$) del gas y determine si el gas se 
          encuentra completamente degenerado a temperatura ambiente.
          }
    
    \item{Si consideramos que cada átomo de la red cristalina del 
          metal contribuye con un solo electrón libre, obtener la relación 
          que hay entre el calor específico del gas de electrones y el calor 
          específico asociado a los grados de libertad vibracionales del 
          sólido a temperatura ambiente (es válido suponer que el metal 
          responde a la ley de Dulong y Petit a esta temperatura).
          }

\end{enumerate}


\newpage
\section{Gas de Bosones}

Consideremos un gas de bosones no interactuantes dentro de un 
recipiente de volumen $V$.
Dada la naturaleza de los bosones, debemos tratar al gas bajo una 
estadística de Bose-Einstein, a partir de la cual podemos determinar el 
número medio de ocupación de cada nivel:

$$ n(\epsilon) = \frac{1}{e^{\beta(\epsilon - \mu)} - 1} $$

\noindent donde $\mu$ es el potencial químico del gas, el cual satisface la 
siguiente relación:

$$ \mu < \epsilon $$

\noindent para todas las energías $\epsilon$ accesibles por el sistema.

\begin{enumerate}[label=(\alph*),
                  leftmargin=2\parindent,
                  rightmargin=2\parindent]

     \item{Justifique por qué es necesario incluir explícitamente el número 
           de ocupación del nivel fundamental ($N_0$) a la hora de calcular el 
           número de partículas en el sistema de la siguiente manera:
           $$ 
           N = 
           N_0 + \int\limits_0^\infty g(\epsilon) n(\epsilon) 
           d\epsilon.
           $$
           Obtenga una expresión de $N_0$, determine la relación entre él y el 
           potencial químico $\mu$ en el caso de que el nivel fundamental 
           posea alta ocupación.
           Demuestre que en dicha situación, el potencial químico es muy 
           próximo a cero.
           }

     \item{Dado que hemos incluido la cantidad de partículas ($N_0$) ocupando 
           el nivel fundamental a la hora de calcular la cantidad total $N$ de 
           partículas en el gas, podemos escribir esta última como:
           $$ N = N_0 + N_e, $$
           donde $N_e$ es la cantidad de partículas ocupando los niveles 
           excitados.
           Obtenga una expresión de $N_e$ en función de la temperatura en el 
           límite de bajas temperaturas.}
     
     \item{Es esperable que a medida que aumentamos la temperatura, la 
           cantidad de bosones en el nivel fundamental vaya mermando.
           Podemos definir entonces una temperatura característica $T_c$ a 
           partir de la cual ningún boson ocupa el nivel fundamental, por 
           ende la cantidad total de partículas se encuentran en los estados 
           excitados:
           $$ N_e(T_c) = N. $$
           Obtenga una expresión de $T_c$ en función de la densidad del gas 
           y demuestre que:
           $$ 
           N_0 =
           N \left[ 1 - \left( \frac{T}{T_c} \right)^\frac{3}{2} \right]
           $$
           Grafique $N_0/N$ en función de $T$.
           Inteprete qué sucede para temperaturas menores a $T_c$.
           }

     \item{Determine el calor específico del gas de bosones a bajas 
           temperaturas en función de $T$, $N$ y $T_c$.
           Teniendo en cuenta que a altas temperaturas el calor específico 
           del gas coincide con el del gas ideal, realice una gráfica 
           esquemática del calor específico en todo el rango de temperaturas.
           }

\end{enumerate}


\section{Paramagnetismo de Pauli}

En prácticas anteriores hemos estudiado el origen del paramagnetismo que 
presentan ciertos materiales debido a la excitación de los momentos 
magnéticos intrínsecos de los átomos de la red cristalina.
Sin embargo existe otro tipo de paramagnetismo que presentan algunos metales 
alcalinos y nobles, el cual no posee su origen en los momentos magnéticos de 
los átomos localizados, sino en la interacción del campo magnético externo 
con los momentos magnéticos intrínsecos de los electrones de conducción del 
metal.

Consideremos un metal compuesto por $N$ átomos del mismo tipo, donde cada uno 
de ellos contribuye con un solo electrón libre al gas de electrones de 
conducción.
Una hipótesis válida es considerar que los electrones de conducción 
constituyen un gas ideal que no interactúa con los átomos de la red ni entre 
ellos.
Supongamos además que este metal se encuentra a temperatura ambiente y es 
sometido a un campo magnético externo débil $H$.
Dado que cada electrón interactúa únicamente con este campo magnético 
externo, los niveles energéticos monoparticulares pueden escribirse como:

$$
\epsilon =
\frac{\hbar^2 \pi^2}{2m_e V^{2/3}} (n_x^2 + n_y^2 + n_z^2) \pm \mu_B H,
\quad
n_x, n_y, n_z = 1, 2, 3, \dots
$$

\noindent donde $V$ es el volumen que ocupa el metal y el signo del término 
asociado a la energía de interacción con el campo viene dado por la 
orientación del spin del electrón: up o down.


\begin{enumerate}[label=(\alph*),
                  leftmargin=2\parindent,
                  rightmargin=2\parindent]

     \item{Si suponemos que dentro del metal hay un electrón por cada
           $10^{-30}$m$^3$, utilice la expresión de la energía de Fermi 
           obtenida en el problema \ref{sec:gas-de-fermi} para determinar el 
           valor de $\epsilon_F$ de los electrones de conducción.
           Obtenga la temperatura de Fermi del gas.
           ¿Podemos considerarlo completamente degenerado?
           }

     \item{Llamemos $N_+$ y $N_-$ a la cantidad de electrones con spin 
           orientado en contra y a favor del campo magnético (down y up) 
           respectivamente.
           Si definimos $g_+(\epsilon)$ y $g_-(\epsilon)$ a las densidades de 
           estado correspondientes a cada grupo de electrones, podemos 
           expresar a $N_\pm$ como:
           $$
           N_\pm =
           \frac{1}{2} \int\limits_{\pm \mu_B H}^{\infty}
           g_\pm(\epsilon) n(\epsilon) d\epsilon.
           $$
           Demuestre que las densidades de estado $g_\pm(\epsilon)$ pueden 
           expresarse en función de la densidad de estado del gas sin campo 
           magnético de la siguiente manera:
           $$ g_\pm(\epsilon) = g(\epsilon \mp \mu_B H). $$
           }

     \item{La magnetización del metal debido a los electrones de conducción 
           puede expresarse como:
           $$ M = \mu_B (N_- - N_+). $$
           Teniendo en cuenta que el campo magnético es débil, obtenga una 
           expresión para dicha magnetización en función del campo $H$.
           Interprete este resultado en función del modelo microscópico 
           (intente hacerlo a través de las gráficas de
           $g(\epsilon \mp \mu_B H)$).
           }

     {\small \textbf{Ayuda:} Dado que el campo magnético es débil, es válido 
     realizar un desarrollo en series de Taylor de la densidad de estados 
     $g(\epsilon \mp \mu_B H)$ en primer orden alrededor de $\epsilon$ a la 
     hora de resolver las integrales de $N_\pm$.
     }

     \item{Teniendo en cuenta el orden de magnitud de la temperatura de Fermi 
           obtenida en el primer item, ¿cómo espera que la magnetización $M$ 
           dependa de la temperatura (para valores de $T$ próximos a la 
           temperatura ambiente, es decir, para temperaturas por debajo del 
           punto de fusión del metal)?
           }

\end{enumerate}


\end{document}

