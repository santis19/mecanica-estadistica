\documentclass[a4paper,11pt]{article}
\usepackage[utf8]{inputenc}
\usepackage[spanish]{babel}
\usepackage[hmargin=3cm, vmargin=3cm]{geometry}
\usepackage{graphicx}
\usepackage{amssymb}
\usepackage{amsmath}
\usepackage{braket}
\usepackage{enumitem}
\usepackage{subcaption}
\usepackage{fancyhdr}
\usepackage{titlesec}


\titleformat{\section}
  {\bf}{Problema \thesection.}{0.5em}{}


%%%%%%%%%%%%%%%%%%%%%%%%%%%%%%%%%%%%%%%%%%%%%%%%%%%%%%%%%%%%%%%%%%%%%%%%%%%%%%

\begin{document}


% Fancy Header
% ------------
\pagestyle{fancy}
%~ \renewcommand{\headrulewidth}{0pt}
\lhead{\small Veronica Gargiulo}
\chead{\small \the\year}
\rhead{\small Santiago Soler}



% Title
% -----
\thispagestyle{plain}
\begin{center}
    \textbf{\large
        Mecánica Estadística \\
        Práctica 4 - Gases Cuánticos
    }
\end{center}
\vspace{-1.5em}



% Excersises
% ----------

\section{Gas de Fermiones}

Consideremos un gas de fermiones no interactuantes dentro de un 
recipiente de volumen $V$.
Dada la naturaleza de los fermiones, debemos tratar al gas bajo una 
estadística de Fermi-Dirac, a partir de la cual podemos determinar el 
número medio de ocupación de cada nivel:

$$ n(\epsilon) = \frac{1}{e^{\beta(\epsilon - \mu)} + 1} $$

\noindent y la gran función de partición:

$$
\mathcal{Q}_{FD} =
  \prod_j \left[ e^{-\beta(\epsilon_j - \mu)} + 1\right],
$$

\noindent donde $\mu$ es el potencial químico del gas.

En el límite de $T = 0$ todos los estados de energía entre 0 y $\mu$ 
están ocupados, mientras que los niveles superiores se encuentran 
disponibles.
En esta situación se considera que el gas está \emph{completamente 
degenerado}.
Dado este estado particular que presentan los gases de fermiones, se 
define a la energía de Fermi $\epsilon_F$ como:

$$ \epsilon_F = \mu(T=0). $$

\begin{enumerate}[label=(\alph*),
                  leftmargin=2\parindent,
                  rightmargin=2\parindent]
    
    \item{Determine el valor medio de número de partículas en el gas 
          ($N$) en el caso de un gas de Fermi completamente degenerado.
          Obtenga una expresión de la energía de Fermi en función de 
          $N$ y $V$, y demuestre que la densidad de estados del gas 
          puede escribirse como:
          $$
          g(\epsilon) =
            \frac{3}{2} \frac{N}{\epsilon_F} \left( 
            \frac{\epsilon}{\epsilon_F} \right)^{\frac{1}{2}}
          $$
          }

    \item{Determine el valor medio de la energía interna ($E$) para el 
          mismo gas completamente degenerado.
          }
    
    \item{Obtenga una expresión de la presión del gas de Fermi completamente 
          degenerado en función de la energía de Fermi.
          Demuestre que, al igual que en el caso del gas ideal 
          clásico, se cumple la siguiente relación entre la presión 
          y la energía interna:
          $$ pV = \frac{2}{3} \langle E \rangle. $$
          Resaltar el hecho de que incluso a $T = 0$, el gas 
          de Fermi presenta presiones no nulas.
          }

    \item{\textbf{Calor específico.}\\
          Para determinar el calor específico del gas de Fermi, al 
          menos en el rango de muy bajas temperaturas, es necesario 
          salirnos del límite de $T = 0$ para así poder obtener una 
          expresión de la energía media en función de $T$.
          La forma correcta de hacerlo sería resolver la integral que 
          involucra la energía interna considerando la expresión analítica 
          del número medio de ocupación $n(\epsilon)$, método que se 
          conoce como el desarrollo de Sommerfeld (1928).
          Sin embargo esto requiere mucho trabajo y se escapa de los 
          objetivos del ejercicio.

          Nosotros intentaremos una forma alternativa\footnote{Este 
          método alternativo se puede encontrar en el libro 
          \emph{Introduction to Solid State Physics}, Kittel, p.151.}, 
          la cual nos permitirá obtener al menos el primer término del 
          desarrollo de Sommerfeld.
          
          Si definimos $\Delta E = E(T) - E(T=0)$, el calor específico 
          puede calcularse como:
          $$
          C_V =
          \left( \frac{\partial \Delta E}{\partial T} \right)_{N, V},
          $$
          ya que restarle la energía del estado fundamental es 
          equivalente a un corrimiento rígido de la energía.
          
          Suponiendo que la cantidad de partículas en el gas no varía 
          significativamente para temperaturas cercanas al cero 
          absoluto, la diferencia entre $N(T)$ y $N(T=0)$ es 
          aproximadamente nula, y por ende puede ser introducida en la 
          expresión de $\Delta E$ (siendo previamente multiplicada por 
          $\epsilon_F$).
          Esta nueva expresión de $\Delta E$ es posible derivarla con 
          respecto a la temperatura para obtener una expresión del 
          calor específico.
          
          Demostrar entonces, que a través de este método obtenemos 
          el primer término del desarrollo de Sommerfeld:
          $$ C_V \simeq N k \frac{\pi^2}{2} \frac{kT}{\epsilon_F}. $$
          
         }

\end{enumerate}

\end{document}

