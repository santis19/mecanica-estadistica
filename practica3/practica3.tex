\documentclass[a4paper,11pt]{article}
\usepackage[utf8]{inputenc}
\usepackage[spanish]{babel}
\usepackage[hmargin=3cm, vmargin=3cm]{geometry}
\usepackage{graphicx}
\usepackage{amssymb}
\usepackage{amsmath}
\usepackage{braket}
\usepackage{enumitem}
\usepackage{subcaption}
\usepackage{fancyhdr}
\usepackage{titlesec}


\titleformat{\section}
  {\bf}{Problema \thesection.}{0.5em}{}


%%%%%%%%%%%%%%%%%%%%%%%%%%%%%%%%%%%%%%%%%%%%%%%%%%%%%%%%%%%%%%%%%%%%%%%%%%%%%%

\begin{document}


% Fancy Header
% ------------
\pagestyle{fancy}
%~ \renewcommand{\headrulewidth}{0pt}
\lhead{\small Veronica Gargiulo}
\chead{\small \the\year}
\rhead{\small Santiago Soler}



% Title
% -----
\thispagestyle{plain}
\begin{center}
    \textbf{\large
        Mecánica Estadística \\
        Práctica 3 - Colectivo Macrocanónico
    }
\end{center}
\vspace{-1.5em}



% Excersises
% ----------

\section{Adsorción}

Consideremos un sólido que se encuentra en equilibrio termodinámico con un 
gas encerrado en un volumen $V$.
La superficie del sólido posee $M$ vacancias que permiten que las partículas 
del gas sean adsorbidas. Vamos a suponer que cada partícula del gas que se 
sitúa en una vacancia posee una energía $-\epsilon_0$ con respecto a la mínima 
que tendría si perteneciese al gas.

\begin{enumerate}[label=(\alph*),
                  leftmargin=2\parindent,
                  rightmargin=2\parindent]

    \item{Determinar la función de partición macrocanónica de las 
          partículas que son adsorbidas por la superficie del sólido.}

    {\small
    \textbf{Ayuda:}
    El teorema del binomio enuncia que dados $p$ y $q$ números reales y $n$ un
    natural, la $n$-ésima potencia del binomio de $p$ y $q$ puede expresarse
    de la siguiente manera:
    $$ (p + q)^n = \sum_{k=0}^n \frac{n!}{(n-k)! \, k!} \, p^k q^{n-k} $$
    }

    \item{\label{item:cubrimiento}
          Calcular la cantidad de partículas adsorbidas $N_0$ en función de 
          $T$ y $\mu$.
          Definamos el \emph{cubrimiento} de la superficie del sólido como 
          $\theta = N_0/M$, es decir, la cantidad de partículas adsorbidas 
          sobre la cantidad de sitios en la superficie.
          Grafique $\theta$ en función de $x = (\epsilon_0 + \mu)/kT$, para 
          valores de $x$ tanto positivos como negativos, ya que el potencial 
          químico $\mu$ puede ser tanto positivo como negativo.}
    
    \item{Demostrar que el potencial químico puede escribirse 
          como:
          $$
          \mu =
          k_B T \ln \left( \frac{\theta}{1 - \theta} \right) - \epsilon_0.
          $$
          }
    
    \item{Modelice el gas como un gas ideal en el colectivo macrocanónico.
          Determine el valor medio de la cantidad de partículas en el gas y 
          despejar de la expresión obtenida el potencial químico del gas.
          }
    
    \item{Dado que el gas y el sólido se encuentran en equilibrio 
          termodinámico, los potenciales químicos de ambos son iguales. 
          Demuestre que la presión de equilibrio del gas puede escribirse 
          como:
          $$ P = \frac{\theta}{1 - \theta} f(T), $$
          donde $f(T)$ es una función de la temperatura.
          }
    
    \item{Si consideramos que el gas posee una densidad muy baja, el volumen 
          medio por partícula es mucho mayor que la longitud de onda térmica 
          de De Broglie, por ende:
          $$ \lambda^3 \frac{N}{V} \ll 1, $$
          donde $N$ es la cantidad de partículas en el gas.
          Determine el signo del potencial químico del gas e interprete la 
          gráfica realizada en el item \ref{item:cubrimiento}. }

\end{enumerate}



\section{Variación de la presión atmosférica con la altitud}

Consideremos un modelo aproximado de la atmósfera dividiendo una 
columna vertical en múltiples capas, donde cada una contiene partículas 
de un gas ideal que se encuentran en equilibrio térmico y difusivo con 
las capas vecinas. Además, por encontrarse en cercanía de la superficie 
de la Tierra, cada partícula se encuentra sometida al potencial 
gravitatorio que esta genera.


\begin{enumerate}[label=(\alph*),
                  leftmargin=2\parindent,
                  rightmargin=2\parindent]

    \item{Obtener la siguiente expresión para el potencial químico de 
          una capa ubicada a una altitud $h$:
          $$ \mu =
          k_B T \ln \left( \frac{\langle N \rangle \lambda^3}{V} \right)
          + mgh $$
          donde $\langle N \rangle$ es el valor medio de la cantidad de 
          partículas en la capa, $V$ es el volumen de la misma, 
          $\lambda$ es la longitud de onda de De Broglie, $m$ es la 
          masa de cada partícula del gas ideal y $g$ es la aceleración 
          de la gravedad en la cercanías de la superficie terrestre.
          }

    {\small
    \textbf{Ayuda:}
    La función exponencial puede expresarse en desarrollo de serie de 
    Taylor de la siguiente manera:
    $$ e^x = \sum_{n=0}^\infty \frac{x^n}{n!} $$
    Y la longitud de onda de De Broglie se define como:
    $$ \lambda = \frac{h}{\sqrt{2\pi m k_B T}} $$
    }

    \item{Obtener la siguiente relación entre la presión atmosférica 
          en función de la altura $h$:
          $$ p(h) = p(0) e^{-h/h_c} $$
          donde $h_c = k_B T / mg$ es una altitud característica.
          Realizar una gráfica de la presión en función de $h/h_c$.
          }
    
    \item{Interpretar el significado de $h_c$ y calcule su valor si 
          consideramos una atmósfera a $T = 290 K$ compuesta por 
          nitrógeno (N$_2$).}

    {\small
    \textbf{Ayuda:}
    El peso atómico del Nitrógeno es igual a 14 g/mol.
    }

    \item{Considerando el valor de $h_c$ obtenido en el punto anterior 
          y una presión de 101325 Pa en la superficie terrestre,
          calcular la presión a 3000m.}

    \item{Discuta la validez de la hipótesis de equilibrio térmico 
          entre las diferentes capas y su relevancia en los resultados 
          obtenidos en los puntos anteriores.}
    
    {\small
    \textbf{Ayuda:}
    La temperatura de la atmósfera desciende (en promedio) 6$^\circ$C 
    por km.
    }

\end{enumerate}

\end{document}

