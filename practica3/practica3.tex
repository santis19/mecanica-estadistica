\documentclass[a4paper,11pt]{article}
\usepackage[utf8]{inputenc}
\usepackage[spanish]{babel}
\usepackage[hmargin=3cm, vmargin=3cm]{geometry}
\usepackage{graphicx}
\usepackage{amssymb}
\usepackage{amsmath}
\usepackage{braket}
\usepackage{enumitem}
\usepackage{subcaption}
\usepackage{fancyhdr}
\usepackage{titlesec}


\titleformat{\section}
  {\bf}{Problema \thesection.}{0.5em}{}


%%%%%%%%%%%%%%%%%%%%%%%%%%%%%%%%%%%%%%%%%%%%%%%%%%%%%%%%%%%%%%%%%%%%%%%%%%%%%%

\begin{document}


% Fancy Header
% ------------
\pagestyle{fancy}
%~ \renewcommand{\headrulewidth}{0pt}
\lhead{\small Veronica Gargiulo}
\chead{\small \the\year}
\rhead{\small Santiago Soler}



% Title
% -----
\thispagestyle{plain}
\begin{center}
    \textbf{\large
        Mecánica Estadística \\
        Práctica 3 - Colectivo Macrocanónico
    }
\end{center}
\vspace{-1.5em}



% Excersises
% ----------

\section{Equilibrio Sólido - Vapor}

Consideremos un sólido encerrado en un recipiente al vacío que 
contiene tanto al sólido como a un vapor de partículas que lo componen.
La superficie del sólido posee $M$ vacancias que permiten que las 
partículas del vapor sean adsorbidas. Vamos a suponer que cada 
partícula del gas que se sitúa en una vacancia posee una energía 
$-\epsilon_0$ con respecto a la que tendría si perteneciese al vapor, 
y que ambas componentes se encuentran en equilibrio 
térmico y químico.

\begin{enumerate}[label=(\alph*),
                  leftmargin=2\parindent,
                  rightmargin=2\parindent]

    \item{Determinar la función de partición macrocanónica de las 
          partículas que son adsorbidas por la superficie del sólido.}

    \item{Calcular la cantidad de partículas adsorbidas $N_0$ en función de 
          $T$ y $\mu$.}
          
    {\small
    \textbf{Ayuda:}
    El teorema del binomio enuncia que dados $p$, $q$ y $n$ números 
    naturales:
    $$ (p + q)^n = \sum_{k=0}^n \frac{n!}{(n-k)! \, k!} \, p^k q^{n-k} $$
    }
    
    \item{Demostrar que el potencial químico puede escribirse 
          como:
          $$ \mu = k_B T \ln(\frac{\theta}{1 - \theta}) - \epsilon_0, $$
          donde $\theta = N_0/M$ es el \emph{cubrimiento}, es decir, la 
          relación entre la cantidad de partículas adsorbidas y la cantidad 
          total de sitios en la superficie del sólido.
          }
    
    \item{Modelice el vapor como un gas ideal en el colectivo macrocanónico.
          Determine el valor medio de la cantidad de partículas en el vapor y 
          despejar de la expresión obtenida el potencial químico del vapor.
          }
    
    \item{Dado que el vapor y el sólido se encuentran en equilibrio 
          termodinámico, los potenciales químicos de ambos son iguales. 
          Demuestre que la presión de equilibrio del gas puede escribirse 
          como:
          $$ P = \frac{\theta}{1 - \theta} f(T), $$
          donde $f(T)$ es una función de la temperatura.
          }

\end{enumerate}



\section{Variación de la presión atmosférica con la altitud}

Consideremos un modelo aproximado de la atmósfera dividiendo una 
columna vertical en múltiples capas, donde cada una contiene partículas 
de un gas ideal que se encuentran en equilibrio térmico y difusivo con 
las capas vecinas. Además, por encontrarse en cercanía de la superficie 
de la Tierra, cada partícula se encuentra sometida al potencial 
gravitatorio que esta genera.


\begin{enumerate}[label=(\alph*),
                  leftmargin=2\parindent,
                  rightmargin=2\parindent]

    \item{Obtener la siguiente expresión para el potencial químico de 
          una capa ubicada a una altitud $z$:
          $$ \mu =
          k_B T \ln \left( \frac{\langle N \rangle \lambda^3}{V} \right)
          + mgz $$
          donde $\langle N \rangle$ es el valor medio de la cantidad de 
          partículas en la capa, $V$ es el volumen de la misma, 
          $\lambda$ es la longitud de onda de De Broglie, $m$ es la 
          masa de cada partícula del gas ideal y $g$ es la aceleración 
          de la gravedad en la cercanías de la superficie terrestre.
          }

    {\small
    \textbf{Ayuda:}
    La función exponencial puede expresarse en desarrollo de serie de 
    Taylor de la siguiente manera:
    $$ e^x = \sum_{n=0}^\infty \frac{x^n}{n!} $$
    Y la longitud de onda de De Broglie se define como:
    $$ \lambda = \frac{h}{\sqrt{2\pi m k_B T}} $$
    }

    \item{Obtener la siguiente relación entre la presión atmosférica 
          en función de la altura $z$:
          $$ p(z) = p(0) e^{-z/z_c} $$
          donde $z_c = k_B T / mg$ es una altitud característica.
          Realizar una gráfica de la presión en función de $z/z_c$.
          }
    
    \item{Interpretar el significado de $z_c$ y calcule su valor si 
          consideramos una atmósfera a $T = 290 K$ compuesta por 
          nitrógeno (N$_2$).}

    {\small
    \textbf{Ayuda:}
    El peso atómico del Nitrógeno es igual a 14 g/mol.
    }

    \item{Considerando el valor de $z_c$ obtenido en el punto anterior 
          y una presión de 101325 Pa en la superficie terrestre,
          calcular la presión a 3000m.}

    \item{Discuta la validez de la hipótesis de equilibrio térmico 
          entre las diferentes capas y su relevancia en los resultados 
          obtenidos en los puntos anteriores.}
    
    {\small
    \textbf{Ayuda:}
    La temperatura de la atmósfera desciende (en promedio) 6$^\circ$C 
    por km.
    }

\end{enumerate}

\end{document}

