\documentclass[a4paper,11pt]{article}
\usepackage[utf8]{inputenc}
\usepackage[spanish]{babel}
\usepackage[hmargin=3cm, vmargin=3cm]{geometry}
\usepackage{graphicx}
\usepackage{amssymb}
\usepackage{amsmath}
\usepackage{braket}
\usepackage{enumitem}
\usepackage{subcaption}
\usepackage{fancyhdr}
\usepackage{titlesec}


\titleformat{\section}
  {\bf}{Problema \thesection.}{0.5em}{}


%%%%%%%%%%%%%%%%%%%%%%%%%%%%%%%%%%%%%%%%%%%%%%%%%%%%%%%%%%%%%%%%%%%%%%%%%%%%%%

\begin{document}


% Fancy Header
% ------------
\pagestyle{fancy}
%~ \renewcommand{\headrulewidth}{0pt}
\lhead{\small Veronica Gargiulo}
\chead{\small \the\year}
\rhead{\small Santiago Soler}



% Title
% -----
\thispagestyle{plain}
\begin{center}
    \textbf{\large
        Mecánica Estadística \\
        Práctica 3 - Colectivo Macrocanónico
    }
\end{center}
\vspace{-1.5em}



% Excersises
% ----------

\section{Interfaz Sólido - Vapor}

Consideremos un sólido encerrado en un recipiente al vacío que 
contiene tanto al sólido como a un vapor de partículas que lo componen.
La superficie del sólido posee $M$ vacancias que permiten que las 
partículas del vapor sean adsorbidas. Vamos a suponer que cada 
partícula del gas que se sitúa en una vacancia posee una energía 
$-\epsilon_0$ con respecto a la que tendría si perteneciese al vapor, 
y que ambas componentes se encuentran en equilibrio 
térmico y químico.

\begin{enumerate}[label=(\alph*),
                  leftmargin=2\parindent,
                  rightmargin=2\parindent]

    \item{Determinar la función de partición macrocanónica de las 
          partículas que son adsorbidas por la superficie del sólido.}

    \item{Calcular la cantidad de partículas adsorbidas $N$ en función de 
          $T$ y $\mu$.}
          
    {\small
    \textbf{Ayuda:}
    El teorema del binomio enuncia que dados $p$, $q$ y $n$ números 
    naturales:
    $$ (p + q)^n = \sum_{k=0}^n \frac{n!}{(n-k)! \, k!} \, p^k q^{n-k} $$
    }
    
    \item{Suponiendo que la superficie tiene una baja ocupación ($N/M 
          \ll 1$), mostrar que el potencial químico puede escribirse 
          como:
          $$ \mu = k_B T \ln(N/M) - \epsilon_0 $$
          }

\end{enumerate}


\end{document}

