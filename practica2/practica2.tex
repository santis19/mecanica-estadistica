\documentclass[a4paper,11pt]{article}
\usepackage[utf8]{inputenc}
\usepackage[spanish]{babel}
\usepackage[hmargin=3cm, vmargin=3cm]{geometry}
\usepackage{graphicx}
\usepackage{amssymb}
\usepackage{amsmath}
\usepackage{enumitem}
\usepackage{subcaption}
\usepackage{fancyhdr}

% Define counter and exercise titles
\newcounter{exercise}
\newcommand{\exercisetitle}[1]{
    \stepcounter{exercise}
    \vspace{1.5em}
    \noindent{
        \textbf{Problema \theexercise . #1}
        \vspace{0.5em}
    }
}


%%%%%%%%%%%%%%%%%%%%%%%%%%%%%%%%%%%%%%%%%%%%%%%%%%%%%%%%%%%%%%%%%%%%%%%%%%%%%%

\begin{document}


% Fancy Header
% ------------
\pagestyle{fancy}
%~ \renewcommand{\headrulewidth}{0pt}
\lhead{\small Veronica Gargiulo}
\chead{\small \the\year}
\rhead{\small Santiago Soler}



% Title
% -----
\thispagestyle{plain}
\begin{center}
    \textbf{\large
        Mecánica Estadística \\
        Práctica 2 - Colectivo Canónico
    }
\end{center}
\vspace{-1.5em}



% Excersises
% ----------

\exercisetitle{Gas ideal clásico}

La forma más sencilla de modelizar un gas real es el \emph{gas ideal 
clásico}, el cual consiste en considerar un conjunto de $N$ partículas 
puntuales cuyas únicas interacciones son colisiones 
perfectamente elásticas.
El caracter clásico del modelo hace referencia a que la dinámica de 
las $N$ partículas puede ser abordada desde la mecánica clásica.

En este problema vamos a considerar un gas ideal dentro de un 
recipiente de volumen $V$ y en contacto con un baño térmico de 
temperatura $T$, de tal manera que el recipiente permite intercambiar 
energía con el baño, y el sistema ``baño + recipiente'' se encuentra 
aislado.

\begin{enumerate}[label=(\alph*),
                  leftmargin=2\parindent,
                  rightmargin=2\parindent]

    \item{\label{item:gas-ideal-particion}
          Si todas las partículas del gas poseen la misma masa $m$, 
          mostrar que la función de partición del gas ideal puede ser:
          $$ Z(N, T, V) = V^N \left[ 2\pi m k_B T \right]^{3N/2} $$
          }

    \item{Calcular la energía libre de Helmholtz y deducir la ecuación de
          estado del gas ideal:
          $$ pV = N k_B T $$
          }
    
    \item{Calcular el valor medio de la energía del sistema y 
          verificar si se cumple el teorema de equipartición.
          }

    \item{Calcular la entropía y los calores específicos $C_V$ y $C_p$.
          }
    
    \item{¿Es la entropía obtenida en el punto anterior una propiedad 
          extensiva?
          }
    
    {\small
    \textbf{Ayuda:} Verificar que la suma de las entropías de dos 
    gases $A$ y $B$ es igual a la entropía del sistema $A + B$.
    }
    
    \item{Agregar la corrección $1/N!$ a la función de partición y 
          verificar si la entropía obtenida a partir de ella es extensiva.
          $$ Z(N, T, V) = \frac{V^N}{N!} \left[ 2\pi m k_B T \right]^{3N/2} $$
          ¿Cómo podemos interpretar la corrección $1/N!$ en función 
          de la distinguibilidad de las partículas clásicas?
          }

\end{enumerate}


\exercisetitle{Longitud de onda de de Broglie}

Consideremos el gas ideal clásico del Problema 1 y repensemos el 
modelo de partículas clásicas no interactuantes. Para que esta 
simplificación sea válida es necesario basarnos en la hipótesis de 
que las partículas del gas se encuentran a una distancia 
suficientemente grande en la que las funciones de onda de 
cada una no se superpone con las del resto, es decir, a una distancia 
en la cual podemos obviar la coherencia cuántica.

Al calcular la función de partición hemos integrado el factor 
$e^{-\beta E}$ en el espacio de las fases. Sin embargo no hemos tenido 
en cuenta un concepto de la mecánica cuántica: el principio de 
incertidumbre de Heisenberg. El mismo dicta que nos es imposible medir 
con precisión ciertos pares de magnitudes físicas de un sistema 
cuántico, por ejemplo, la posición y el momento. Este principio se 
puede resumir en:
$$ \Delta x \Delta p \gtrsim  h$$
donde $\Delta x$ y $\Delta p$ son las incertidumbres en la medición de 
la posición y el momento de una particula, respectivamente; y $h$ es 
la constante de Planck.
De esta manera, podríamos pensar que a la hora de integrar en las 
coordenadas del espacio de las fases es necesario dividir por el 
``mínimo volumen'' que puede ocupar cada microestado 
$\{ q_1, \dots q_N, p_1, \dots p_N \}$, es decir, $h^{3N}$.

\begin{enumerate}[label=(\alph*),
                  leftmargin=2\parindent,
                  rightmargin=2\parindent]
      
    \item{Obtener la función de partición a partir de la siguiente 
          integral:
          $$
          Z(N, T, V) =
            \frac{1}{h^{3N} N!}
            \int_D e^{-\beta E(\{q, p\})} d^{3N}q \,\, d^{3N}p
          $$
          }

    \item{Definimos a la longitud de onda térmica de de Broglie como:
          $$ \lambda = \left( \frac{h^2}{2\pi m k_B T} \right)^{1/2} $$
          Expresar la función de partición obtenida en el punto 
          anterior en función de $\lambda$.
          }

    \item{La longitud de onda de de Broglie se puede considerar como la 
          longitud característica a partir de la cual la coherencia 
          cuántica comienza a tener relevancia.\\
          Obtener una expresión para la distancia media de las 
          partículas ($l$). ¿Cómo deberíamos considerar a las 
          partículas si $\lambda/l \ll 1$? ¿y en el caso contrario?
          }
          
\end{enumerate}


\end{document}

